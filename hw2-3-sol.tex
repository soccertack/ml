\documentclass[11pt]{article}
\usepackage[letterpaper,margin=1in]{geometry}
\usepackage{listings}
\usepackage{url}
\usepackage{hyperref}
\usepackage{textcomp}
\usepackage{graphicx}
\usepackage{underscore}
\lstset{%
  basicstyle=\small\ttfamily,
  mathescape=true,
  upquote=true,
}

\usepackage{amsmath,amsbsy,amsfonts,amssymb,amsthm,color,dsfont,mleftright,commath}

\def\ddefloop#1{\ifx\ddefloop#1\else\ddef{#1}\expandafter\ddefloop\fi}

% \bbA, \bbB, ...
\def\ddef#1{\expandafter\def\csname bb#1\endcsname{\ensuremath{\mathbb{#1}}}}
\ddefloop ABCDEFGHIJKLMNOPQRSTUVWXYZ\ddefloop

% \cA, \cB, ...
\def\ddef#1{\expandafter\def\csname c#1\endcsname{\ensuremath{\mathcal{#1}}}}
\ddefloop ABCDEFGHIJKLMNOPQRSTUVWXYZ\ddefloop

% \vA, \vB, ..., \va, \vb, ...
\def\ddef#1{\expandafter\def\csname v#1\endcsname{\ensuremath{\boldsymbol{#1}}}}
\ddefloop ABCDEFGHIJKLMNOPQRSTUVWXYZabcdefghijklmnopqrstuvwxyz\ddefloop

% \valpha, \vbeta, ...,  \vGamma, \vDelta, ...,
\def\ddef#1{\expandafter\def\csname v#1\endcsname{\ensuremath{\boldsymbol{\csname #1\endcsname}}}}
\ddefloop {alpha}{beta}{gamma}{delta}{epsilon}{varepsilon}{zeta}{eta}{theta}{vartheta}{iota}{kappa}{lambda}{mu}{nu}{xi}{pi}{varpi}{rho}{varrho}{sigma}{varsigma}{tau}{upsilon}{phi}{varphi}{chi}{psi}{omega}{Gamma}{Delta}{Theta}{Lambda}{Xi}{Pi}{Sigma}{varSigma}{Upsilon}{Phi}{Psi}{Omega}{ell}\ddefloop

\newcommand\braces[1]{\{#1\}}

\theoremstyle{definition}
\newtheorem{problem}{Problem}
\newenvironment{solution}{\noindent\emph{Solution.}}{\hfill$\square$}

%-------------------------------------------------------------------------------

\title{COMS 4771 Spring 2017 Homework 2}
\author{Jin Tack Lim, jl4312
  }
\date{%
  }

\begin{document}
\maketitle


%-------------------------------------------------------------------------------
% \frac{1}{{ \sqrt {2\pi \sigma ^2} }}e^{{{ - \left( {x - \mu } \right)^2 } \mathord{\left/ {\vphantom {{ - \left( {x - \mu } \right)^2 } {2\sigma ^2 }}} \right. \kern-\nulldelimiterspace} {2\sigma ^2 }}}

\newcommand{\gau}[3]{
$\frac{1}{{ \sqrt {2\pi #3 ^2} }}e^{ \frac { - \left( {#1 - #2 } \right)^2 } {2#3 ^2}}$
}

\newcommand{\gauraw}[3]{
\frac{1}{{ \sqrt {2\pi #3 ^2} }}e^{ \frac { - \left( {#1 - #2 } \right)^2 } {2#3 ^2}}
}


Problem 3

(a) Exponential distribution

[Posterior]

The given prior is the gamma distribution with $\alpha = 4$ and $\beta = 1$
\begin{equation*}
\begin{split}
 p(\alpha | X ) & =  \frac{p(\alpha)p(X|\alpha)}{p(X)} \\
 & = \frac{\frac{\lambda^3 e^{-\lambda}}{6} \prod_{i=1}^{N} \lambda e^{-\lambda x_i}} {p(X)} \\
 & \propto {\lambda^{N+3} e^{-\lambda} e^{\sum_{i=1}^{N} x_i}} \\
 & = {\lambda^{(N+4) -1} e^{-\lambda(1+\sum_{i=1}^{N} x_i)} } 
\end{split}
\end{equation*}
Therefore

\begin{equation*}
\begin{split}
 p(\alpha | X ) & =  Gamma(\lambda; N+4, 1+\sum_{i=1}^{N} x_i)  \\
& = \frac{(1+\sum_{i=1}^{n} x_i)^{N+4}}{\Gamma(N+4)} \lambda^{N+3} e^{-\lambda(1+\sum_{i=1}^{n} x_i)}
\end{split}
\end{equation*}


[EAP]

Mean of the gamma distribution $E[Gamma(\lambda;\alpha, \beta)] = \frac{\alpha}{\beta}$

Therefore
\begin{equation*}
\begin{split}
EAP = \frac{N+4}{1+\sum_{i=1}^{N} x_i}
\end{split}
\end{equation*}

(b) Coin problem

[Posterior]

Let's say $N_1$ is the number of heads out of N trials.
\begin{equation*}
\begin{split}
 p(\alpha | X ) & =  \frac{p(\alpha)p(X|\alpha)}{p(X)}  \\
& = \frac{30\alpha^2(1-\alpha)^2\alpha^{N_1} (1-\alpha)^{N-{N_1}}}{p(X)}  \\
& = \frac{30\alpha^{N_1+2} (1-\alpha)^{N-{N_1}+2}}{p(X)}  \\
& = \frac{\alpha^{N_1+2} (1-\alpha)^{N-{N_1}+2}}{c(N_1 +2, N - N_1 +2)} 
\end{split}
\end{equation*}
where c(m, k) is defined as following
\begin{equation*}
\begin{split}
c(m, k) = \frac{m!k!}{(m+k+1)!} 
\end{split}
\end{equation*}

[EAP]
\begin{equation*}
\begin{split}
EAP & = E[Posterior]  \\
& = E[\frac{\alpha^{N_1+2} (1-\alpha)^{N-{N_1}+2}}{c(N_1 +2, N - N_1 +2)}]  \\
& = \frac{\int_{0}^{1}\alpha \alpha^{N_1+2} (1-\alpha)^{N-{N_1}+2} d\alpha}{c(N_1 +2, N - N_1 +2)}  \\
& = \frac{c(N_1 +3, N - N_1 +2)}{c(N_1 +2, N - N_1 +2)}  \\
& = \frac{N_1 + 3}{N + 6}
\end{split}
\end{equation*}

References

[1] https://www.probabilitycourse.com/chapter9/9_1_1_prior_and_posterior.php

[2] Bishop 2.3

[3] https://en.wikipedia.org/wiki/Gamma_function

[4] http://web.cse.ohio-state.edu/~kulis/teaching/788_sp12/scribe_notes/lecture3.pdf
\end{document}

