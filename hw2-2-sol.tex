\documentclass[11pt]{article}
\usepackage[letterpaper,margin=1in]{geometry}
\usepackage{listings}
\usepackage{url}
\usepackage{hyperref}
\usepackage{textcomp}
\usepackage{graphicx}
\usepackage{underscore}
\lstset{%
  basicstyle=\small\ttfamily,
  mathescape=true,
  upquote=true,
}

\usepackage{amsmath,amsbsy,amsfonts,amssymb,amsthm,color,dsfont,mleftright,commath}

\def\ddefloop#1{\ifx\ddefloop#1\else\ddef{#1}\expandafter\ddefloop\fi}

% \bbA, \bbB, ...
\def\ddef#1{\expandafter\def\csname bb#1\endcsname{\ensuremath{\mathbb{#1}}}}
\ddefloop ABCDEFGHIJKLMNOPQRSTUVWXYZ\ddefloop

% \cA, \cB, ...
\def\ddef#1{\expandafter\def\csname c#1\endcsname{\ensuremath{\mathcal{#1}}}}
\ddefloop ABCDEFGHIJKLMNOPQRSTUVWXYZ\ddefloop

% \vA, \vB, ..., \va, \vb, ...
\def\ddef#1{\expandafter\def\csname v#1\endcsname{\ensuremath{\boldsymbol{#1}}}}
\ddefloop ABCDEFGHIJKLMNOPQRSTUVWXYZabcdefghijklmnopqrstuvwxyz\ddefloop

% \valpha, \vbeta, ...,  \vGamma, \vDelta, ...,
\def\ddef#1{\expandafter\def\csname v#1\endcsname{\ensuremath{\boldsymbol{\csname #1\endcsname}}}}
\ddefloop {alpha}{beta}{gamma}{delta}{epsilon}{varepsilon}{zeta}{eta}{theta}{vartheta}{iota}{kappa}{lambda}{mu}{nu}{xi}{pi}{varpi}{rho}{varrho}{sigma}{varsigma}{tau}{upsilon}{phi}{varphi}{chi}{psi}{omega}{Gamma}{Delta}{Theta}{Lambda}{Xi}{Pi}{Sigma}{varSigma}{Upsilon}{Phi}{Psi}{Omega}{ell}\ddefloop

\newcommand\braces[1]{\{#1\}}

\theoremstyle{definition}
\newtheorem{problem}{Problem}
\newenvironment{solution}{\noindent\emph{Solution.}}{\hfill$\square$}

%-------------------------------------------------------------------------------

\title{COMS 4771 Spring 2017 Homework 2}
\author{Jin Tack Lim, jl4312
  }
\date{%
  }

\begin{document}
\maketitle


%-------------------------------------------------------------------------------
% \frac{1}{{ \sqrt {2\pi \sigma ^2} }}e^{{{ - \left( {x - \mu } \right)^2 } \mathord{\left/ {\vphantom {{ - \left( {x - \mu } \right)^2 } {2\sigma ^2 }}} \right. \kern-\nulldelimiterspace} {2\sigma ^2 }}}

\newcommand{\gau}[3]{
$\frac{1}{{ \sqrt {2\pi #3 ^2} }}e^{ \frac { - \left( {#1 - #2 } \right)^2 } {2#3 ^2}}$
}

\newcommand{\gauraw}[3]{
\frac{1}{{ \sqrt {2\pi #3 ^2} }}e^{ \frac { - \left( {#1 - #2 } \right)^2 } {2#3 ^2}}
}


Problem 2

(a) Posteriori:
\begin{eqnarray}
 p(\alpha | X ) \propto p(\alpha)p(X|\alpha) \nonumber 
\end{eqnarray}
\begin{eqnarray}
 = \gauraw{\mu}{\mu_p}{\sigma_p} \prod_{i=1}^{N} \gauraw{x_i}{\mu}{\sigma} \nonumber
\end{eqnarray}
(Remove/Add constant terms to make the following term)
\begin{eqnarray}
 \propto exp\Bigg\{ - \frac {\mu - \frac{\mu_p\sigma^2 + \sum_{i=1}^{N} \sigma_p^2x_i}{\sigma^2 + N\sigma_p^2}}
		{2 \frac{\sigma_p^2\sigma^2}{\sigma^2 + N\sigma_p^2}} \Bigg\} \nonumber
\end{eqnarray}

Once we define
\begin{eqnarray}
\sigma_N = \frac{\sigma_p^2\sigma^2}{\sigma^2 + N\sigma_p^2} \nonumber \\
\mu_N = \frac{\mu_p\sigma^2 + \sum_{i=1}^{N} \sigma_p^2x_i}{\sigma^2 + N\sigma_p^2} \nonumber
\end{eqnarray}

then,
\begin{eqnarray}
 p(\alpha | X ) \propto exp{\frac{-(\mu - \mu_N)^2}{2\sigma_N^2}} \nonumber 
\end{eqnarray}

We can apply the Gaussian integral, and get the posteriori.
\begin{eqnarray}
 p(\alpha | X ) = \gauraw{\mu}{\mu_N}{\sigma_N} \nonumber
\end{eqnarray}

(b) the maximum-a-posteriori

We can simply take the log of the posteriori, and take the derivative and set it zero to get the maximum-a-posteriori.


\begin{eqnarray}
\mu_{MAP} = \mu_N = \frac{\mu_p\sigma^2 + \sum_{i=1}^{N} \sigma_p^2x_i}{\sigma^2 + N\sigma_p^2} \nonumber
\end{eqnarray}

(c) expected-a-posteriori value

Given that the posteriori is the Gaussian distribution, we can easily tell that the expected value is the mean. 
\begin{eqnarray}
E[posteriori] = \mu_N = \frac{\mu_p\sigma^2 + \sum_{i=1}^{N} \sigma_p^2x_i}{\sigma^2 + N\sigma_p^2} \nonumber
\end{eqnarray}

References

[1] https://www.probabilitycourse.com/chapter9/9_1_1_prior_and_posterior.php

[2] Bishop 2.3

\end{document}


