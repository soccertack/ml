\documentclass[11pt]{article}
\usepackage[letterpaper,margin=1in]{geometry}
\usepackage{listings}
\usepackage{url}
\usepackage{hyperref}
\usepackage{textcomp}
\usepackage{graphicx}
\usepackage{underscore}
\lstset{%
  basicstyle=\small\ttfamily,
  mathescape=true,
  upquote=true,
}

\usepackage{amsmath,amsbsy,amsfonts,amssymb,amsthm,color,dsfont,mleftright,commath}

\def\ddefloop#1{\ifx\ddefloop#1\else\ddef{#1}\expandafter\ddefloop\fi}

% \bbA, \bbB, ...
\def\ddef#1{\expandafter\def\csname bb#1\endcsname{\ensuremath{\mathbb{#1}}}}
\ddefloop ABCDEFGHIJKLMNOPQRSTUVWXYZ\ddefloop

% \cA, \cB, ...
\def\ddef#1{\expandafter\def\csname c#1\endcsname{\ensuremath{\mathcal{#1}}}}
\ddefloop ABCDEFGHIJKLMNOPQRSTUVWXYZ\ddefloop

% \vA, \vB, ..., \va, \vb, ...
\def\ddef#1{\expandafter\def\csname v#1\endcsname{\ensuremath{\boldsymbol{#1}}}}
\ddefloop ABCDEFGHIJKLMNOPQRSTUVWXYZabcdefghijklmnopqrstuvwxyz\ddefloop

% \valpha, \vbeta, ...,  \vGamma, \vDelta, ...,
\def\ddef#1{\expandafter\def\csname v#1\endcsname{\ensuremath{\boldsymbol{\csname #1\endcsname}}}}
\ddefloop {alpha}{beta}{gamma}{delta}{epsilon}{varepsilon}{zeta}{eta}{theta}{vartheta}{iota}{kappa}{lambda}{mu}{nu}{xi}{pi}{varpi}{rho}{varrho}{sigma}{varsigma}{tau}{upsilon}{phi}{varphi}{chi}{psi}{omega}{Gamma}{Delta}{Theta}{Lambda}{Xi}{Pi}{Sigma}{varSigma}{Upsilon}{Phi}{Psi}{Omega}{ell}\ddefloop

\newcommand\braces[1]{\{#1\}}

\theoremstyle{definition}
\newtheorem{problem}{Problem}
\newenvironment{solution}{\noindent\emph{Solution.}}{\hfill$\square$}

%-------------------------------------------------------------------------------

\title{COMS 4771 Spring 2017 Homework 3}
\author{Jin Tack Lim, jl4312
  }
\date{%
  }

\begin{document}
\maketitle


%-------------------------------------------------------------------------------
% \frac{1}{{ \sqrt {2\pi \sigma ^2} }}e^{{{ - \left( {x - \mu } \right)^2 } \mathord{\left/ {\vphantom {{ - \left( {x - \mu } \right)^2 } {2\sigma ^2 }}} \right. \kern-\nulldelimiterspace} {2\sigma ^2 }}}

\newcommand{\gau}[3]{
$\frac{1}{{ \sqrt {2\pi #3 ^2} }}e^{ \frac { - \left( {#1 - #2 } \right)^2 } {2#3 ^2}}$
}

\newcommand{\gauraw}[3]{
\frac{1}{{ \sqrt {2\pi #3 ^2} }}e^{ \frac { - \left( {#1 - #2 } \right)^2 } {2#3 ^2}}
}


Problem 2

(a)

Constraint: $\sum_{i=d}^{D} x_d = C$

If we take derivatives for each dimension d,

$\frac{d}{x_d} = \lambda$

We can plug this to the Constraint equation, and get

$\frac{1}{\lambda} = \frac {2C}{D(D+1)}$

Therefore

$ x_d = \frac{2Cd}{D(D+1)}$

and Maximum value of f(x) is

$$\boxed{\sum_{d=1}^{D}d log \frac{2Cd}{D(D+1)}}$$


(b)

Constraint: $\sum_{d=1}^{D} x_i^2 \leq 1$

So, these are all equations and inequalities we have.

$ \frac {1}{x_d} = \lambda 2x_d $ for each dimension d.

$\sum_{d=1}^{D} x_d^2 \leq 1$

$\lambda (\sum_{i=d}^{D} x_d^2 -1 ) = 0 $

From the first equation, if $\lambda$ is zero, then all $x_i$ goes to infinity which does not satisfy the second inequality. So $\sum_{d=1}^{D} x_d^2 =1$.

Then, by solving above equations,

$x_d = \sqrt{\frac{2d}{D(D+1)}}$

and Maximum value of f(x) is

$$\boxed{\sum_{d=1}^{D}d log \sqrt{\frac{2d}{D(D+1)}}}$$


\end{document}


