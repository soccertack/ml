\documentclass[11pt]{article}
\usepackage[letterpaper,margin=1in]{geometry}
\usepackage{listings}
\usepackage{url}
\usepackage{hyperref}
\usepackage{textcomp}
\usepackage{graphicx}
\usepackage{underscore}
\lstset{%
  basicstyle=\small\ttfamily,
  mathescape=true,
  upquote=true,
}

\usepackage{amsmath,amsbsy,amsfonts,amssymb,amsthm,color,dsfont,mleftright,commath}

\def\ddefloop#1{\ifx\ddefloop#1\else\ddef{#1}\expandafter\ddefloop\fi}

% \bbA, \bbB, ...
\def\ddef#1{\expandafter\def\csname bb#1\endcsname{\ensuremath{\mathbb{#1}}}}
\ddefloop ABCDEFGHIJKLMNOPQRSTUVWXYZ\ddefloop

% \cA, \cB, ...
\def\ddef#1{\expandafter\def\csname c#1\endcsname{\ensuremath{\mathcal{#1}}}}
\ddefloop ABCDEFGHIJKLMNOPQRSTUVWXYZ\ddefloop

% \vA, \vB, ..., \va, \vb, ...
\def\ddef#1{\expandafter\def\csname v#1\endcsname{\ensuremath{\boldsymbol{#1}}}}
\ddefloop ABCDEFGHIJKLMNOPQRSTUVWXYZabcdefghijklmnopqrstuvwxyz\ddefloop

% \valpha, \vbeta, ...,  \vGamma, \vDelta, ...,
\def\ddef#1{\expandafter\def\csname v#1\endcsname{\ensuremath{\boldsymbol{\csname #1\endcsname}}}}
\ddefloop {alpha}{beta}{gamma}{delta}{epsilon}{varepsilon}{zeta}{eta}{theta}{vartheta}{iota}{kappa}{lambda}{mu}{nu}{xi}{pi}{varpi}{rho}{varrho}{sigma}{varsigma}{tau}{upsilon}{phi}{varphi}{chi}{psi}{omega}{Gamma}{Delta}{Theta}{Lambda}{Xi}{Pi}{Sigma}{varSigma}{Upsilon}{Phi}{Psi}{Omega}{ell}\ddefloop

\newcommand\braces[1]{\{#1\}}

\theoremstyle{definition}
\newtheorem{problem}{Problem}
\newenvironment{solution}{\noindent\emph{Solution.}}{\hfill$\square$}

%-------------------------------------------------------------------------------

\title{COMS 4771 Spring 2017 Homework 1}
\author{Jin Tack Lim, jl4312
  }
\date{%
  }

\begin{document}
\maketitle


%-------------------------------------------------------------------------------

Problem 3.

This is Poisson distribution as discussed in the first lecture.
\bigbreak

Loss function = $-\log \frac{{e^{ - f(x) } f(x) ^y }}{{y!}}$ where x is tumor size, $f(x) = \lambda$ and y is the time from the patient’s admission till them passing away.
\bigbreak

When we have n samples, log likely hood is 
\bigbreak

$ \log \prod_{i=1}^{\infty} \frac{{e^{ - f(x) } f(x) ^y }}{{y!}}$
\bigbreak

and this is maximized when $ f(x) = \lambda  = \frac{x_1 + x_2 + ... + x_n}{n}$
\bigbreak

Empirical risk R = $ \frac{1}{n} \sum_{i=1}^{n} -\log \frac{{e^{ - f(x_i) } f(x_i) ^{y_i} }}{{y_i!}} $
\bigbreak

\end{document}

